\documentclass{homework}
\usepackage{homework}
\title{report}
\subtitle{week 3}
\begin{document}

\maketitle
\section{isValid.c}
\subsection{函数说明}
\paragraph{输入}
指向一个字符串的指针
\paragraph{输出}
一个Bool值:该字符串中三种括号‘(’、‘)’,‘\lbrack’、‘\rbrack’和‘\{’、‘\}’是否符合规则
\paragraph{实现方法}

以数组实现栈。依次遍历字符串,

遇到左括号则压栈对应数字‘(’:1,‘\lbrack’:2,‘\{’:3);

遇到右括号则检查栈顶是否为对应的左括号:

\hspace{2em}若对应则弹栈;

\hspace{2em}若不对应则返回false;

结束后若栈空则返回true,若不为空则返回false。
\subsection{复杂度}
\paragraph{时间复杂度}
$\mathcal{O}(n)$
\paragraph{空间复杂度}
$\mathcal{O}(1)$
\subsection{边界情况}

当输入为空字符串时没有进行压栈,最后判断栈为空,输出true;

当字符串长度超过设定的MAXLEN时会出现栈溢出。

若采用头插法的链表来实现栈,可以将空间复杂度从$\mathcal{O}(1)$“降”为$\mathcal{O}(n)$,此时栈溢出取决于运行电脑的内存,而不是设定的MAXLEN。但由于无法在函数内部声明链表结构体,只得用更为简单的数组来实现栈。
\subsection{程序运行结果}

...

Score:12 / 12
\section{getIntersectionNode.c}
\subsection{函数说明}
\paragraph{输入}
两个链表的头节点
\paragraph{输出}
两个链表合并的节点
\paragraph{实现方法}

先求出两个链表的长度,然后将两个链表后端对齐,从短链表的头节点和长链表的对应节点开始比较,直到出现相同节点,即为合并点。

注意只需对比节点本身(地址)即可,不同节点可能具有相同的值。
\subsection{复杂度}
\paragraph{时间复杂度}
$\mathcal{O}(n)$
\paragraph{空间复杂度}
$\mathcal{O}(1)$
\subsection{边界情况}

存在空链表时,对比过程不进行,直接进入最后返回NULL。

若两链表无合并,则知道对比结束也不会有匹配,返回NULL。
\subsection{程序运行结果}

...

Correct answer.

Start testing with 10000 random cases...

Your code runs well!
\end{document}
